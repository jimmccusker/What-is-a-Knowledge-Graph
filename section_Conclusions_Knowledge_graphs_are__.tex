\section{Conclusions}

Knowledge graphs are a critical component of the Semantic Web and serve as information hubs for general use as well as for domain-specific applications.
Most knowledge graphs seek to aggregate knowledge from third party sources, whether from external databases, crawling the Web, or using entity and relationship extraction techiniques.
Knowledge graphs are not simply aggregations of RDF or linked data, but instead specifically focus on time-invariant information about entities of general interest.
They tend to rely on a limited set of relations and try to adhere to a coherent knowledge model.
This sets them apart from the linked data cloud in general, which usually has relied on the open framework of the Semantic Web to provide completely free-form use of vocabularies and ontologies.
While some knowledge graphs track the provenance of their content, it is by no means a universal practice.
We argue that, instead, knowledge graphs should always provide the epistemology (how we know what we know) of the knowledge it contains, and that Nanopublications are a suitable framework in which to do so.
Semantic publishing that does not provide a level of statement epistemology can be considered ``Bare Statement'' graphs.
Since so many knowledge graphs are curated from third parties, and because of the nature of publishing on the Web (Anyone can say Anything about Any subject), as knowledge graphs increase in popularity it will become critical to avoid use of such ``Bare Statement'' graphs.