\section{Conclusions}

Knowledge graphs are a critical component of the Semantic Web and serve as information hubs for general use as well as for domain-specific applications.
Most knowledge graphs seek to aggregate knowledge from third party sources, whether from external databases, from data aggregated though crawling the Web, or through the application of entity and relationship extraction methods.
Knowledge graphs are not simply aggregations of RDF or linked data, but critically provide time-invariant information about entities of general interest.
Their structures tend to be focused on a limited set of relations adhering to a coherent knowledge model, setting them apart from the linked data cloud in general, which usually has relied on the open framework of the Semantic Web to accommodate a completely free-form use of vocabularies and ontologies.
Although some knowledge graphs track the provenance of their content, rigorous provenance is by no means a universal characteristic.
We argue that knowledge graphs should prioritize the epistemology of the knowledge it contains -- how we know what we know -- and that Nanopublications are a suitable framework in which to do so.
Semantic publishing that does not provide a level of statement epistemology can be considered ``Bare Statement'' graphs.
Since so many knowledge graphs are curated from third parties, and because of the nature of publishing on the Web (\textit{Anyone} can say \textit{Anything} about \textit{Any} subject), as knowledge graphs increase in popularity it will become critical to avoid use of such ``Bare Statement'' graphs.