\section{Discussion}


\subsection{Future Potential}

Currently, knowledge graphs are not usually distinguished from what we refer to as ``bare statement'' graphs, in that they do not encode or publish the epistemology (why something is known) of the knowledge in the graph.
This is troubling because it does not privilege knowledge in knowledge graphs.
Unsupported assertions are given equal weight in current knowledge graphs.
Instead, there is an opportunity to leverage existing vocabularies, like the Provenance Ontology (PROV-O) \cite{Moreau_2015}, and the nanopublications framework \cite{groth2010anatomy}.
A nanopublication is a set of RDF graphs: an assertion graph (the knowledge), a provenance graph (the justification), and an attribution graph (the believer).
While justified true belief is not sufficient for knowledge, most other proposals, such as a causal linkage between the justification, assertion, and believer, are well supported within provenance vocabularies.
The provenance graph can then expand to provide room for whatever epistemic criteria is desired.

There is an interesting overlap between what is considered a ``knowledge graph'' and what is an ontology.
The most commonly accepted definition of an ontology is a ``an explicit specification of a conceptualization'' \cite{Gruber_1993}.
To a large degree, knowledge graphs conform to this definition, but generally ontologies tend to talk about generalities (classes, properties, and roles) instead of specific instances.
Few ontologies would mention the Eiffel Tower, but a knowledge graph that covers the domain of Parisian landmarks would.
Conversely, knowledge graph approaches can be used to improve the credibility of ontologies by encoding the epistemology of the statements in the ontology.
