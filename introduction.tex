\section{Introduction}


% Classic intro vs abstract issues here.
Knowledge graphs provide an opportunity to expand our understanding of how knowledge can be managed on the web and how that knowledge can be distinguished from more conventional web-based data publication schemes such as Linked Data \cite{bizer2009linked}.
In recent years knowledge graphs have grown increasingly prominent through commercial and research applications on the web.
Google was one of the first to promote something explicitly identified as a ``knowledge graph,'' \cite{singhal2012introducing} and many other organizations have since followed the term both in the literature and in less formal communication.
We review the last known formal definition of knowledge graphs, knowledge graph analysis and construction algorithms, and popular commercial and research knowledge graphs in the literature.
These new knowledge graphs are not strictly adhering to original knowledge graph theory, but instead have followed a looser, more operationally convenient definition.
We will in turn present a more descriptive view of what knowledge graphs currently are, and also discuss what they can become in the future.