\section{Introduction}


% Classic intro vs abstract issues here.
Knowledge graphs provide an opportunity to expand our understanding of how knowledge can be managed on the Web and how that knowledge can be distinguished from more conventional Web-based data publication schemes such as Linked Data \cite{bizer2009linked}.
In recent years knowledge graphs have grown increasingly prominent through commercial and research applications on the Web.
Google was one of the first to promote a semantic metadata organizational model described as a ``knowledge graph,'' \cite{singhal2012introducing} and many other organizations have since used the term in the literature and in less formal communication.
Our purpose with this paper is to provide an explicit description of the evolving notion of a knowledge graph, and further to lay out a potential impact spectrum.  
We review recent formal definition of knowledge graphs, knowledge graph analysis and construction algorithms, and popular commercial and research knowledge graphs in the literature.
These new knowledge graphs do not strictly adhere to original knowledge graph theory \cite{van1992knowledge}, but instead have followed a looser, more flexible definition.
We present a more descriptive view of current, practical knowledge graphs, and discuss their potential for evolution and impact.