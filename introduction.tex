\section{Introduction}


% Classic intro vs abstract issues here.
Knowledge graphs provide an opportunity to expand the notions of how knowledge can be managed on the web and how that knowledge is distinguished from more conventional data publication schemes like Linked Data \cite{bizer2009linked}.
Knowledge graphs have become more prominent in commercial and research applications on the web.
Google was one of the first to promote something explicitly labeled as a ``knowledge graph,'' \cite{singhal2012introducing} and many other organizations have since followed the term both in the literature and in less formal communication.
We will review the last known formal definition of knowledge graphs, knowledge graph analysis and construction algorithms, and popular commercial and research knowledge graphs in the literature.
These new knowledge graphs are not strictly adhering to original knowledge graph theory, but instead have followed a looser, more operationally convenient definition.
We will in turn present a more descriptive view of what knowledge graphs currently are, and also discuss what they can become in the future.