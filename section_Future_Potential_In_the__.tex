\section{Future Potential}

In the literature knowledge graphs are not (usually) distinguished from bare statement graphs, in that they do not encode or publish the epistemology \footnote{Epistemology defines why something is known} of knowledge asserted in the graph.
We see this as troubling because it does not {\em privilege} knowledge: in most existing knowledge graphs supported and unsupported assertions are given equal weight.
Moving forward, there is an opportunity to leverage existing vocabularies, including the Provenance Ontology (PROV-O) \cite{Moreau_2015}, and the Nanopublications Framework \cite{groth2010anatomy}, to improve the clarity and utility of knowledge graphs.
A nanopublication is a set of RDF graphs: an {\em assertion graph} (the knowledge), a {\em provenance graph} (the justification), and an {\em attribution graph} (the believer).
While justified true belief is not sufficient for knowledge, most other proposals, including a causal linkage between the justification, assertion, and believer, are well-supported within provenance vocabularies.
Added to a knowledge graph, the provenance graph can expand to provide room for whatever epistemic criteria is desired.

There is an interesting overlap between what is considered a ``knowledge graph'' and what is an ontology.
The most commonly accepted definition of an ontology is a ``an explicit specification of a conceptualization'' \cite{Gruber_1993}.
To a large degree, knowledge graphs conform to this definition, but generally ontologies tend to talk about generalities (classes, properties, and roles) with less focus on inclusion of content about specific instances.
For example, most ontologies that include content related to descriptions of world landmarks would have descriptions of the landmark class and its related properties but would typically not include a mention of the Eiffel Tower, but a knowledge graph that covers the domain of Parisian landmarks would.
Conversely, knowledge graph approaches can be used to improve the credibility of ontologies by encoding the epistemology of the statements in the ontology.


Ontology vs Knowledge Graph vs Data Graph?