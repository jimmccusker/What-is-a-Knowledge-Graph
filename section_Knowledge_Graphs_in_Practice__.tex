\section{Knowledge Graphs in Practice}
Rospocher, et al. define knowledge graphs as collections of time-invariant facts about entities, typically derived from structured data sources such as Freebase \cite{Rospocher2016}. They cite a dearth of event representations in current knowledge graphs as a shortcoming - limiting knowledge graphs to encyclopedic items such as birth and death dates - primarily due to the difficulty of obtaining temporal data about entities in a structured manner. Recent surveys such as those by Hogenboom, et al. \cite{Hogenboom2016} and Deng, et al. \cite{Deng2015} provide overviews of numerous methods for event extraction from a variety of sources including social media, news, academic publications, and even images and video, indicating that there is a great interest in finding ways to interpret and include such temporal data in a more structured format.

This is the literature review. Start here: \url{https://scholar.google.com/scholar?hl=en&q=knowledge+graph}