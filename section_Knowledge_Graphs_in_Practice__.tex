\section{Knowledge Graphs in Practice}
Rospocher, \textit{et al.} present knowledge graphs as collections of typically time-invariant facts about entities, typically derived from structured data sources such as Freebase and \cite{Rospocher2016}. They cite a dearth of event representations in current knowledge graphs as a shortcoming - limiting knowledge graphs to encyclopedic items such as birth and death dates - primarily due to the difficulty of obtaining temporal data about entities in a structured manner. Recent surveys such as those by Hogenboom, \textit{et al.} \cite{Hogenboom2016} and Deng, \textit{et al.} \cite{Deng2015} provide overviews of numerous methods for event extraction from a variety of sources including social media, news, academic publications, and even images and video, indicating that there is a great interest in finding ways to interpret and include such temporal data in a more structured format.
Another review by Nickel \emph{et al.} explores machine learning methods for knowledge graphs, but limits their definition to directed labeled graphs, with the ability to optionally pre-define the schema.
They also review but do not take a position on the use of the closed versus open world assumptions.

van de Riet and Meersman \cite{van1992knowledge}, Stokman and de Vries \cite{Stokman_1988}, and Zhang \cite{zhang2002knowledge}, present a formal theory of knowledge graphs as a specialization of semantic networks where meaning is expressed as structure, statements are unambiguous, and a limited set of relation types are used.
These requirements also minimize redundancy within the knowledge graph, which simplifies analytical operations (including reasoning and queries).
Popping explores the use of knowledge graphs and their challenges at the time in their use in network text analysis \cite{Popping_2003}. 
Following Zhang, Popping defines the knowledge graph as a type of semantic network that uses only a few types of relations, but also asserts that additional knowledge may be added to the graph.

More papers to consider: \cite{Dieng_1992} \cite{Juel_Vang_2013}

\subsection{Knowledge Graph Methods}
Corby and Zucker present an abstract knowledge graph querying machine they call KGRAM \cite{Corby_2010}, but do not define knowledge graphs beyond being labeled directed graphs.
This seems to be an abstraction of graph query methods and discusses how KGRAM is a generalization and extension of the RDF graph query language SPARQL \cite{harris2013sparql}.
Wang \emph{et al.} \cite{Wang_knowledgegraph} discuss projecting generalized knowledge graphs into hyperplanes, but also only focuses on the labeled directed graph requirement of knowledge graphs.
Pujara \emph{et al.} use probabilistic soft logic (PSL) to manage uncertainty in knowledge graphs that have been extracted from uncertain sources \cite{Pujara_2013}. 
They argue that many current knowledge graphs do not always clearly identify entities, relying instead on labels that can be different due to spelling variations.
Their task of ``knowledge graph identification'' has a goal of identifying a set of true assertions from noisy extractions.
They do not claim to manage the provenance of the resulting knowledge graph assertions, however.
Lin \emph{et al.} attempt link prediction for automated knowledge graph construction but only rely on a directed labeled graph model of knowledge graphs \cite{lin2015learning}.
Hakkani-Tur \emph{et al.} use statistical language understanding to pose structured questions against the Freebase knowledge graph, focusing on improving the extraction of relation detection in the queries \cite{Hakkani_Tur_2013}.
Benedek \emph{et al.} have presendted a collaborative knowledge graph construction tool called  ``Conceptipedia'', building off of their ``WikiNizer'' project \cire{benedek}.
This project uses visual mind mapping techniques and concept similarity analysis to suggest cross-knowledge graph mappings between collaborators.
Weiderman and Kritzinger \cite{} refer to knowledge graphs as a synonym for concept maps, but do not expand further on the topic, nor do they cite any work in knowledge graphs.
 

\subsection{Academic Knowledge Graphs}
The Gene Ontology (GO), can be considered more of a knowledge graph than an ontology \cite{Ashburner_2000}.
It has created a hierarchy of biological processes, cellular locations, and molecular functions into which a number of genes and proteins have been classified.
These annotations have been curated by domain experts, and the provenance of each is recorded using a GO-specific provenance encoding.
YAGO (Yet Another Great Ontology) \cite{Suchanek_2007} and YAGO 2 \cite{Hoffart_2013} have been considered by some researchers to be knowledge graphs, although they started as a large, general-purpose ontology.
While it aggregates knowledge from many sources, there are no published descriptions of if or how provenance is tracked. 

The XLore system claims to be a fully bilingual (Chinese and English) knowledge graph that focuses on extracting \emph{subClassOf} and \emph{instanceOf} relations from free text \cite{wang2013xlore}.
SEKI@home is a crowd-sourced knowledge graph that is aggregated from multiple sources \cite{steiner2012seki}.
This project maintains entity-level provenance using the PROV Ontology \cite{Moreau_2015}.
The project has also incorporated real-time matching against news articles \cite{steiner_iswc_2012}.
The Knowledge Vault handles knowledge graph uncertainty as a result of automated fact extraction from Web pages \cite{Dong_2014}.
DBPedia is a large-scale transformation of Wikipedia into a knowledge graph \cite{Bizer_2009}.
It uses a mostly fixed schema and provides provenance of which Wikipedia pages each entity was derived from.
A number of biomedical knowledge graphs have been constructed from public databases, including Bio2RDF\cite{Callahan_2013}, Neurocommons \cite{Ruttenberg_2009}, and LinkedLifeData \cite{momtchev2009expanding}.
All three knowledge graphs provide dataset-level provenance.

\subsection{Commercial Knowledge Graphs}
Freebase is a knowledge graph of over 3B facts and 58M topics \footnote{Freebase.com web site, April 2016} that is open to public access and curation \cite{Bollacker_2008} and was the basis for the Google Knowledge Graph, which augmented Freebase with knowledge gleaned from Google's regular search engine crawls of the Web \cite{singhal2012introducing}.
Monteiro and Moura \cite{10110943220141101} present a thoughtful analysis of the role of the Google Knowledge Graph as a realization of the Semantic Web vision \cite{bernerslee2000semantic} as Web 4.0, and show how it merges rule-oriented semantic analysis with statistical predictive approaches.
Microsoft has also introduced a knowledge graph called ``Satori'' to enhance Bing search results \cite{qian2013understand}.
