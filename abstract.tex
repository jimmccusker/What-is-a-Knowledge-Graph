%% Add back in before submission.
%\begin{abstract}
Google introduced its Knowledge Graph project in 2012, and has used it to improve query result relevancy and their overall search experience.
They have leveraged existing knowledge graphs, such as DBpedia and Freebase, and also have opened up the process of contributing to the graph by ingesting RDFa and microdata formats from the web pages they index, based on the vocabularies published by schema.org.
The success of the Google Knowledge Graph, and its use of semantic technologies, has led to a resurgence in the use of the term in semantic research to describe similar projects.
However, the term ``knowledge graph'' remains underspecified, and in many cases, simply refers to any directed labeled graph.
We surveyed both current literature on knowledge graphs and the historical use of the term and found some consistencies between current and historical use.
The pre-semantic web conceptualization of knowledge graphs provides us with guidance as to what might currently ``count'' as a knowledge graph, but also points towards future possible uses for current resources.
We used this overview to establish an updated definition and requirements for knowledge graphs, and also address an implicit requirement: that knowledge graphs represent knowledge, as opposed to bare assertions with no justification or provenance.
We discuss how knowledge graphs as defined are a crucial component of the future of the web and have great potential for transformational change in data science and domain sciences.
%\end{abstract}